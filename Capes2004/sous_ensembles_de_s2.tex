\section{Étude d'un sous ensemble $\mathrm{X}$ de $\mathrm{S}^2$ sans points fixe de $\mathrm{G}$.}
On a vu dans \ref{2.} que l'ensemble $\mathrm{X} \subset \mathrm{S}^2$ qu'on doit construire doit vérifier que l'action de $\mathrm{G}$ sur lui soit libre, ce qui nous amène à éviter les éléments $x\in  \mathrm{S}^2$ qui verifient $g(x) = x$ pour un certain $g \in \mathrm{G}$. On définit l'ensemble $\mathrm{X}$ de la façon suivante :
\begin{definition}
  On considère l'ensemble $$\mathrm{F}=\{v \in \mathrm{S}^2 | \exists g \in \mathrm{G}-\{\mathrm{Id_E}\}\text{, } g(v)=v\}$$ et on définit l'ensemble $\mathrm{X}$ comme étant le complémentaire de $\mathrm{F}$ c.à.d : $$\mathrm{X} = \mathrm{S}^2 \textbackslash \mathrm{F}$$
\end{definition}
\begin{prop}\label{prop6}
  \hfill

  Pour tout élément $g$ dans $\mathrm{G}$, l'application :
  \begin{align*}
 g_X \colon X &\to X\\
 x &\mapsto g(x).
\end{align*}\par
est bijective et :
  \begin{align*}
  \phi \colon \mathrm{G} &\to \mathfrak{S}(X)\\
  g &\mapsto g_X.
\end{align*}\par
est un homomorphisme injectif de groupes.
\end{prop}
\begin{lemma}\label{lemme10}
  L'ensemble $\mathrm{X}$ n'est pas vide, et son complémentaire est dénombrable.
\end{lemma}
\begin{proof}
  \hfill

  On sait que $\mathrm{G}$ est un sous-groupe de $\mathcal{SO}(\mathrm{E})$ et donc pour tout $g$ dans $\mathrm{G}$ il existe une droite vectorielle $D$ \par tel que $g$ est une rotation d'axe $D$.

    %
    \begin{center}
    \begin{tikzpicture}[tdplot_main_coords, scale = 2]
    % Draw shaded circle
    \shade[ball color = lightgray,
        opacity = 0.5
    ] (0,0,0) circle (1cm) ;
    % draw arcs
    \tdplotsetrotatedcoords{0}{0}{0};
    \draw[dashed,
        tdplot_rotated_coords,
        gray
    ] (0,0,0) circle (1);
    \tdplotsetrotatedcoords{90}{90}{90};
    \draw[dashed,
        tdplot_rotated_coords,
        gray
    ] (1,0,0) arc (0:180:1);
    \tdplotsetrotatedcoords{0}{90}{90};
    \draw[dashed,
        tdplot_rotated_coords,
        gray
    ] (1,0,0) arc (0:180:1);
    % Axes in 3 d coordinate system
    \draw[-stealth] (0,0,0) -- (1.90,0,0)
        node[below left] {$x$};
    \draw[-stealth] (0,0,0) -- (0,1.30,0)
        node[below right] {$y$};
    \draw[-stealth] (0,0,0) -- (0,0,1.30)
        node[above] {$z$};
    \draw[dashed, gray] (0,0,0) -- (-1,0,0);
    \draw[dashed, gray] (0,0,0) -- (0,-1,0);
    \draw[thick, color = gray] (3,3,3) -- (-3,-3,-3) node[below left] {$D$};
    \coordinate (P) at ({1/sqrt(3)},{1/sqrt(3)},{1/sqrt(3)});
    \coordinate (P') at ({-1/sqrt(3)},{-1/sqrt(3)},{-1/sqrt(3)});
    \draw[fill = lightgray!50] (P) circle (0.5pt) node[right, color=blue] {$x$};
    \draw[fill = lightgray!50] (P') circle (0.5pt) node[left, color=blue] {$x'$};

    \draw[thin, dashed, color=blue] (P) --++ (0,0,{-1/sqrt(3)});
    \draw[thin, dashed, color=blue] ({1/sqrt(3)},{1/sqrt(3)},0) --++
    (0,{-1/sqrt(3)},0);
    \draw[thin, dashed, color=blue] ({1/sqrt(3)},{1/sqrt(3)},0) --++
    ({-1/sqrt(3)},0,0);
    \draw[thin, dashed, color=blue] (0, 0, {1/sqrt(3)}) --++
    ({1/sqrt(3)}, {1/sqrt(3)}, 0);
    \draw[thin, dashed, color=blue] ({1/sqrt(3)},{1/sqrt(3)},0) --++
    ({-1/sqrt(3)},{-1/sqrt(3)},0);

    \draw[thin, dashed, color=blue] (P') --++ (0,0,{1/sqrt(3)});
    \draw[thin, dashed, color=blue] ({-1/sqrt(3)},{-1/sqrt(3)},0) --++
    (0,{1/sqrt(3)},0);
    \draw[thin, dashed, color=blue] ({-1/sqrt(3)},{-1/sqrt(3)},0) --++
    ({1/sqrt(3)},0,0);
    \draw[thin, dashed, color=blue] (0, 0, {-1/sqrt(3)}) --++
    ({-1/sqrt(3)}, {-1/sqrt(3)}, 0);
    \draw[thin, dashed, color=blue] ({-1/sqrt(3)},{-1/sqrt(3)},0) --++
    ({1/sqrt(3)},{1/sqrt(3)},0);

    %
    \draw[dashed, gray] (0,0,0) -- (-1,0,0);
    \draw[dashed, gray] (0,0,0) -- (0,-1,0);
    \draw[dashed, gray] (0,0,0) -- (0,0,-1);
    %


    \end{tikzpicture}
  \end{center}
    %
    \par
  Donc la droite vectorielle $D$ intersecte $\mathrm{S}^2$ en deux points $x_g$ et $x'_g=-x_g$, alors :
  $$\mathrm{F}=\underset{g \in \mathrm{G}-\{\mathrm{Id_E}\}}{\bigcup}\{x_g, -x_g\} $$
  \par Or $\mathrm{G}$ est dénombrable et $\{x_g, -x_g\}$ est fini donc $\mathrm{F}$ est dénombrable. De plus il est clair que la sphere $\mathrm{S}^2$\par n'est pas dénombrable car si on prend juste un cercle de la sphere par exemple celui pour laquelle $z=0$ \par alors ce dernier est isomorphe à $[-\pi, \pi[$ qui n'est pas dénombrable, et donc l'ensemble $\mathrm{X}$ ne peut pas être \par vide; car sinon $\mathrm{F}=\mathrm{S}^2$, et l'un est dénombrable et l'autre ne l'est pas.\par
\end{proof}
%%%%%%%%%%%%%%%%%%%%%%%%%%%%%%%%%%%%%%%%%%%%%%%%%%%%%%%%%%%%%%%%%%%%%%%%%%%%%%%%%%%%%%%%%%%%%%%%%%%%%%%%%%%%%%%%%%%%%%%%%
\begin{lemma}\label{lemme11}
  L'ensemble $\mathrm{X}$ est stable sous l'action de $\mathrm{G}$, i.e :
  $$\forall g \in \mathrm{G},\forall v \in \mathrm{X}, g(v) \in \mathrm{X}$$
\end{lemma}
\begin{proof}
  \hfill

Soit $v \in \mathrm{X}$ si $g\left(v\right) \in \mathrm{F}$ alors $\exists g' \in \mathrm{G}-\{\mathrm{Id_E}\}, g'\left(g\left(v\right)\right)=g\left(v\right)$ et donc $g^{-1}\left(g'\left(g\left(v\right)\right)\right)=v \in \mathrm{F}$.
\end{proof}
%%%%%%%%%%%%%%%%%%%%%%%%%%%%%%%%%%%%%%%%%%%%%%%%%%%%%%%%%%%%%%%%%%%%%%%%%%%%%%%%%%%%%%%%%%%%%%%%%%%%%%%%%%%%%%%%%%%%%%%%%
\begin{lemma}\label{lemme12}
  Soient $g$ et $h$ deux éléments de $\mathrm{G}$. Si $\exists v \in \mathrm{X}$ vérifiant $g(v)=h(v)$, alors $h=g$.
\end{lemma}
\begin{proof}
  \hfill

  Si $h \neq g$, $h^{-1}\circ g \neq \mathrm{Id_E}$ et $h^{-1}\circ g(v)=v$ alors $v \in \mathrm{F}$ ce qui n'est pas possible car $v \in \mathrm{X}= \mathrm{S}^2-\mathrm{F}$.
\end{proof}
\noindent
Jusqu'ici on a montré que l'ensemble $\mathrm{X}$ est différent de $\emptyset$, et que l'application $g_X \colon X \to X \colon x \to g(x)$ est bien définie $\forall g \in \mathrm{G}$. Donc on va maintenant démontrer la \hyperref[prop6]{Proposition 5}.\par
%%%%%%%%%%%%%%%%%%%%%%%%%%%%%%%%%%%%%%%%%%%%%%%%%%%%%%%%%%%%%%%%%%%%%%%%%%%%%%%%%%%%%%%%%%%%%%%%%%%%%%%%%%%%%%%%%%%%%%%%%
\begin{proof}[Démonstration de la Proposition 5]
  \hfill

  Puisque $\forall g \in \mathrm{G}, \forall v \in \mathrm{E}, \norme{g(v)} = \norme{v}$, la sphere unitaire est stable par $g$, et donc tout élément $g$ induit une\par bijection de $\mathrm{S}^2$. D'après le \hyperref[lemme11]{Lemme 9} $$\forall v \in \mathrm{X}, g(v) \in \mathrm{X}, g^{-1}(v) \in \mathrm{X}$$\par donc $g$ induit une bijection de $\mathrm{X}$ sur lui même que l'on notera $g_X$. \par
  %%%%%%%%%%%%%%%%%%%%%%%%%%%%%%%%%%%%%%%%%%%%%%%%%%%%%%%%%%%%%%%%%%%%%%%%%%%%%%%%%%%%%%%%%%%%%%%%%%%%%%%%%%%%%%%%%%%%%%%%%
  Soit $v \in \mathrm{X}$, alors :
  \begin{align*}
  \phi(g \circ h)(v) &= g\circ h(v)\\
   &= \phi(g) \circ \phi(h)(v) \text{ car $y=h(v)\in \mathrm{X}$ et $g(y) \in \mathrm{X}$}
  \end{align*}
  \par
  de plus $\phi$ est injective, car si $g(v)=h(v), \forall v \in \mathrm{X}$ alors d'après le \hyperref[lemme12]{Lemme 10} $h=g$.\par
\end{proof}

% \begin{lemma}\label{lemme13}
%   La relation $\sim_\mathrm{G}$ définie par : $a\sim_\mathrm{G}b \Leftrightarrow \exists g \in \mathrm{G}, a=g(b)$ est une relation d'équivalence sur $\mathrm{X}$.
% \end{lemma}
% \begin{proof}
%   \hfill
%
%   Il est clair que $a = \mathrm{Id_G}(a)$ donc $a\sim_\mathrm{G}a$ càd $\sim_\mathrm{G}$ est réfléxive, de plus si $a = g(b)$, alors $b = g^{-1}(a)$, et donc \par $\sim_\mathrm{G}$ est symétrique, et enfin si $a=g(b)$ et $b=h(c)$ alors $a=g\circ h(c)$, donc $\sim_\mathrm{G}$ est transitive, et donc $\sim_\mathrm{G}$ est \par bien une relation d'équivalence.\par
% \end{proof}
%%%%%%%%%%%%%%%%%%%%%%%%%%%%%%%%%%%%%%%%%%%%%%%%%%%%%%%%%%%%%%%%%%%%%%%%%%%%%%%%%%%%%%%%%%%%%%%%%%%%%%%%%%%%%%%%%%%%%%%%%
\begin{definition}
  On sait que les orbites forment une partition de $\mathrm{X}$, donc en utilisant \hyperref[axiome]{l'axiome du choix} on définit l'ensemble $\mathrm{M}$ en choisissant un représentant de chaque orbite, pour garantir une unicité des éléments de $X$ après l'action de $G$.
\end{definition}

\begin{lemma}\label{lemme14}
   La famille $(g(\mathrm{M}))_{g \in \mathrm{G}}$ constitue une partition de $\mathrm{X}$.
\end{lemma}
\begin{proof}
  \hfill

On vérifie facillement que
  \begin{itemize}
    \item Si $x \in g(M)\cap g'(M)$ alors $\exists y, y' \in M, x=g(y)=g'(y')$ ce qui n'est pas possible par définition de $M$.
    \item $M \ne \emptyset$ donc $\forall g \in \mathrm{G}, g(M) \ne \emptyset$.
    \item $\underset{g \in \mathrm{G}}{\bigcup}g(M) = \underset{a \in \mathrm{M}}{\bigcup}\underset{g \in \mathrm{G}}{\bigcup}g(a) = \mathrm{X}$
  \end{itemize}\par
  Donc $(g(\mathrm{M}))_{g \in \mathrm{G}}$ est une partition de $\mathrm{X}$. Visuellement si on prend $M = \{a_1, a_2 ...\}$ alors :

  \begin{center}
    $\mathrm{X} = $
  \begin{tabular}{|c|c|c|c|c|}
  \hline
  $a_1$ & $a_2$ & $a_3$ & $.$ & $.$\\
  % \hline
  $g_1(a_1)$ & $g_1(a_2)$ & $g_1(a_3)$ & $.$ & . \\
  % \hline
  $g_2(a1)$ & $g_2(a_2)$ & $g_2(a_3)$ & . & . \\
  % \hline
  . & . & . & . & .  \\
  % \hline
  . &.&.&.&.\\
  \hline
  \end{tabular}
  =
  \begin{tabular}{|r|}
  \hline
  \text{$a_1$ $a_2$ $a_3$ . . . . . . . . . .}\\
  \hline
  $g_1(a_1)$  $g_1(a_2)$  $g_1(a_3)$  $.$  . \\
  \hline
  $g_2(a1)$  $g_2(a_2)$  $g_2(a_3)$  .  . \\
  \hline
  . . . . . . . . . . . . . . . . .  \\
  \hline
  . . . . . . . . . . . . . . . . .\\
  \hline
  \end{tabular}
  \end{center}
  \par
  c'est à dire qu'on peut répartir $\mathrm{X}$ soit horizontalement ou vérticalement.

  \par
\end{proof}
%%%%%%%%%%%%%%%%%%%%%%%%%%%%%%%%%%%%%%%%%%%%%%%%%%%%%%%%%%%%%%%%%%%%%%%%%%%%%%%%%%%%%%%%%%%%%%%%%%%%%%%%%%%%%%%%%%%%%%%%%
\begin{prop}\label{prop7}
  On pose :
  $$X_0 = \mathrm{M}, X_1 = \underset{g \in L(\rho)}{\bigcup} g(\mathrm{M}), X_2 = \underset{g \in L(\tau)}{\bigcup} g(\mathrm{M}), X_3 = \underset{g \in L(\rho^{-1})}{\bigcup} g(\mathrm{M}), X_4 = \underset{g \in L(\tau^{-1})}{\bigcup} g(\mathrm{M})$$\par
  \par Les parties $(X_0, X_1, X_2, X_3, X_4)$, $(X_1, \rho(X_3))$ et $(X_2, \tau(X_4))$ forment des partitions de $\mathrm{X}$.
\end{prop}
\begin{proof}
  \hfill

\noindent
  Puisque $\mathrm{G} = \langle \rho, \tau\rangle = \{ \mathrm{Id_E}\}\cup \mathrm{L}(\rho) \cup \mathrm{L}(\rho^{-1}) \cup \mathrm{L}(\tau) \cup \mathrm{L}(\tau^{-1})$, où $\rho$ et $\tau$ sont des isométries positives, et les réunions sont disjointes, et  $(g(\mathrm{M}))_{g \in \mathrm{G}}$ est une partition de $\mathrm{X}$ alors on conclut que: $(X_0, X_1, X_2, X_3, X_4)$ \text{ est une partition de } $\mathrm{X}$.\\
  On a déjà montré que : $\mathrm{G}=\mathrm{L}(\rho)\cup\rho\mathrm{L}(\rho^{-1})= \mathrm{L}(\tau)\cup\tau\mathrm{L}(\tau^{-1})$, où les réunions sont disjointes,  et  $(g(\mathrm{M}))_{g \in \mathrm{G}}$ est une partition de $\mathrm{X}$ donc il suit que
  \begin{align*}
    \mathrm{X} &= X_1\cup \rho(X_3) \text{ et } X_1 \cap \rho(X_3) = \emptyset \\
    \mathrm{X} &= X_2\cup \tau(X_4) \text{ et } X_2 \cap \tau(X_4) = \emptyset
  \end{align*}
Il suit que $(X_1, \rho(X_3))$ et $(X_2, \tau(X_4))$ forment des partitions de $\mathrm{X}$, et donc $\mathrm{X}$ est \hyperref[ed]{équidécomposable} à deux copies de $\mathrm{X}$.
\end{proof}

\begin{remarkk}
  On remarque que les ensembles construits sont non mésurables (au sens le Lebesgue), et que c'est l'axiome du choix qui nous a permit de construire de tels ensembles.\\
  En effet, si les ensembles étaient mésurable alors on aurait que
  \begin{align*}
    \mu(X) &= \mu(X_0) + \mu(X_1)+ \mu(X_2)+\mu(X_3)+\mu(X_4) = \mu(X_1) + \mu(\rho(X_3)) = \mu(X_2) + \mu(\rho(X_4))\\
    &=\mu(X_1) + \mu(X_3)=\mu(X_2) + \mu(X_4)\text{ (la rotation ne change pas la mésure d'un ensemble mésurable). }\\
    &=0 \text{, ce qui est absurde car $\mu(\mathrm{X}) = \mu(\mathrm{S}^2)$, puisque $F$ est dénombrable.}
  \end{align*}
  où $\mu$ est la mésure de Lebesgue.
\end{remarkk}
\noindent
Dans la partie suivante on va essayer de décomposer $\mathrm{S}^2$ en parties et les réaranger après composition par des rotations et des translations pour construire l'ensemble $\mathrm{X}$. Pour cela on va introduire la notion d'équidécomposabilité.
