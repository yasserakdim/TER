\section{Rotations d'un espace vectoriel euclidien orienté de dimention 3 .}
\begin{prop}\label{prop1}
  Soit $\rho_1$ et $\rho_2$ deux rotations vectorielles $\neq \mathrm{Id_E}$ de $\mathrm{E}$, alors :
  $$\rho_1, \rho_2 \text{ commutent } \Leftrightarrow (\text{$\rho_1$ et $\rho_2$ ont même axe ou sont des demi-tour d'axes orthogonaux})$$
\end{prop}
\begin{proof}
\hfill
\begin{itemize}
  \item On commence par l'implication réciproque, et on suppose dans un premier temps que $\rho_1$ et $\rho_2$ sont des rotations  d'angles $\alpha_1$ et $\alpha_2$ respectivement autour de la droite $D$ dirigée et orientée par un vecteur unitaire $\omega$, qu'on peut le compléter en une base orthonormée $B = (\omega, e, u)$.\\ Les matrices de $\rho_1$ et $\rho_2$ dans cette base sont:
  \[
  \mathrm{P}_1=
  \begin{pmatrix}
  1 & 0 & 0\\
  0 & \cos(\alpha_1) & -\sin(\alpha_1) \\
  0 & \sin(\alpha_1) & \cos(\alpha_1)
  \end{pmatrix}
  \text{ et }
  \mathrm{P}_2=
  \begin{pmatrix}
  1 & 0 & 0\\
  0 & \cos(\alpha_2) & -\sin(\alpha_2) \\
  0 & \sin(\alpha_2) & \cos(\alpha_2)
  \end{pmatrix}
  \]
  Alors il est clair que :
  \[
   \mathrm{P}_1 \mathrm{P}_2=\mathrm{P}_2 \mathrm{P}_1=
   \begin{pmatrix}
   1 & 0 & 0\\
   0 & \cos(\alpha_1+\alpha_2) & -\sin(\alpha_1+\alpha_2) \\
   0 & \sin(\alpha_1+\alpha_2) & \cos(\alpha_1+\alpha_2)
   \end{pmatrix}
   \]
   qui est la rotation d'axe $D$ qui est dirigée et orientée par $\omega$, d'angle $\alpha_1 + \alpha_2$.
  %
  \tdplotsetmaincoords{60}{110}

  \begin{center}
  \begin{tikzpicture}[tdplot_main_coords, scale = 2.5]
  \coordinate (P) at ({1/sqrt(3)},{1/sqrt(3)},{1/sqrt(3)});
  \coordinate (P_1) at ({-1/sqrt(3)},{1/sqrt(3)},{1/sqrt(3)});
  \coordinate (P_2) at (0,{-1/sqrt(3)},{1/sqrt(3)});
  \coordinate (P_3) at ({1/sqrt(3)},0,{1/sqrt(3)});

  \coordinate (O) at (0,0,0);

  \draw[thin, dashed] (P) --++ (0,0,{-1/sqrt(3)});
  \draw[thin, dashed] ({1/sqrt(3)},{1/sqrt(3)},0) --++
  (0,{-1/sqrt(3)},0);
  \draw[thin, dashed] ({1/sqrt(3)},{1/sqrt(3)},0) --++
  ({-1/sqrt(3)},0,0);
  \draw[thin, dashed] (0, 0, {1/sqrt(3)}) --++
  ({1/sqrt(3)}, {1/sqrt(3)}, 0);
  \draw[thin, dashed] ({1/sqrt(3)},{1/sqrt(3)},0) --++
  ({-1/sqrt(3)},{-1/sqrt(3)},0);

  \draw[thin, dashed, color=red] (P_1) --++ (0,0,{-1/sqrt(3)});
  \draw[thin, dashed,color=red] ({-1/sqrt(3)},{1/sqrt(3)},0) --++
  (0,{-1/sqrt(3)},0);
  \draw[thin, dashed, color=red] ({-1/sqrt(3)},{1/sqrt(3)},0) --++
  ({1/sqrt(3)},0,0);
  \draw[thin, dashed, color=red] (0, 0, {1/sqrt(3)}) --++
  ({-1/sqrt(3)}, {1/sqrt(3)}, 0);
  \draw[thin, dashed, color=red] ({-1/sqrt(3)},{1/sqrt(3)},0) --++
  ({1/sqrt(3)},{-1/sqrt(3)},0);

  \draw[thin, dashed, color=blue] (P_2) --++ (0,0,{-1/sqrt(3)});
  \draw[thin, dashed, color=blue] (P_2) --++
  (0,{1/sqrt(3)},0);

  \draw[thin, dashed, color=purple] (P_3) --++ (0,0,{-1/sqrt(3)});
  \draw[thin, dashed, color=purple] (P_3) --++
  ({-1/sqrt(3)},0,0);


  \draw[thick, -stealth, color=red] (0,0,0) -- (P_1) node[right] {$\rho_1(x)$};
  \draw[thick, -stealth, color=blue] (0,0,0) -- (P_2) node[left] {$\rho_2(x)$};
  \draw[thick, -stealth, color=purple] (0,0,0) -- (P_3) node[left] {$\rho_2(\rho_1(x))=\rho_1(\rho_2(x))$};



  % Axes in 3 d coordinate system
  \draw[-stealth] (0,0,0) -- (1.80,0,0)
      node[below left] {$e$};

  \draw[-stealth] (0,0,0) -- (0,1.30,0)
      node[below right] {$u$};

  \draw[-stealth] (0,0,0) -- (0,0,1.30)
      node[above] {$\omega$};
  \draw[thick, -stealth] (0,0,0) -- (P) node[right] {$x$};
  \pgfmathsetmacro{\phivec}{135}
  \tdplotdrawarc[-stealth, color=red]{(O)}{0.5}{45}{\phivec}{anchor=west}{$\alpha_1$}
  \pgfmathsetmacro{\phivec}{-90}
  \tdplotdrawarc[-stealth, color=blue]{(O)}{0.5}{45}{\phivec}{anchor=north}{$\alpha_2$}

  \draw[fill = lightgray!50] (P) circle (0.5pt);
  \draw[fill = lightgray!50] (P_1) circle (0.5pt);
  \draw[fill = lightgray!50] (P_2) circle (0.5pt);

  %
  \draw[dashed, gray] (0,0,0) -- (-1,0,0);
  \draw[dashed, gray] (0,0,0) -- (0,-1,0);
  \draw[dashed, gray] (0,0,0) -- (0,0,-1);
  %
  \end{tikzpicture}
  \end{center}
   %

  \item Si maintenant $\rho_1$ et $\rho_2$ sont deux demi-tour d'axes respectifs $D_1$ et $D_2$ orthogonaux, alors si on prend $e_1$ et $e_2$ deux vecteurs unitaires directeurs de $D_1$ et $D_2$ respectivement
  et on les complète en une base orthonormée $B=(e_1, e_2, e_3)$ on trouve que les matrices de  $\rho_1$ et $\rho_2$ dans $B$ sont :
  \[
  \mathrm{P}_1=
  \begin{pmatrix}
  1 & 0 & 0\\
  0 & -1 & 0 \\
  0 & 0 & -1
  \end{pmatrix}
  \text{ et }
  \mathrm{P}_2=
  \begin{pmatrix}
  -1 & 0 & 0\\
  0 & 1 & 0 \\
  0 & 0 & -1
  \end{pmatrix}
  \]
  alors il est clair que :
   \[
   \mathrm{P}_1 \mathrm{P}_2=\mathrm{P}_2 \mathrm{P}_1=
   \begin{pmatrix}
   -1 & 0 & 0\\
   0 & -1 & 0 \\
   0 & 0 & 1
   \end{pmatrix}
   \]
   qui est la rotation d'axe $D_3$ qui est dirigée et orientée par $e_3$, d'angle $\pi$.
   \begin{center}
   \begin{tikzpicture}[tdplot_main_coords, scale = 2.5]
   % Create a point (P)...
   \coordinate (P) at ({1/sqrt(3)},{1/sqrt(3)},{1/sqrt(3)});
   \coordinate (P_1) at ({1/sqrt(3)},{-1/sqrt(3)},{-1/sqrt(3)});
   \coordinate (P_2) at ({-1/sqrt(3)},{1/sqrt(3)},{-1/sqrt(3)});
   \coordinate (P_3) at ({-1/sqrt(3)},{-1/sqrt(3)},{1/sqrt(3)});
   % Projection of the point on X, Y and Z axis
   \draw[thin, dashed] (P) --++ (0,0,{-1/sqrt(3)});
   \draw[thin, dashed] ({1/sqrt(3)},{1/sqrt(3)},0) --++
   (0,{-1/sqrt(3)},0);
   \draw[thin, dashed] ({1/sqrt(3)},{1/sqrt(3)},0) --++
   ({-1/sqrt(3)},0,0);
   \draw[thin, dashed] (0, 0, {1/sqrt(3)}) --++
   ({1/sqrt(3)}, {1/sqrt(3)}, 0);
   \draw[thin, dashed] ({1/sqrt(3)},{1/sqrt(3)},0) --++
   ({-1/sqrt(3)},{-1/sqrt(3)},0);
   \draw[thin, dashed, color=red] (P_1) --++ (0,0,{1/sqrt(3)});
   \draw[thin, dashed, color=red] ({1/sqrt(3)},{-1/sqrt(3)},0) --++
   (0,{1/sqrt(3)},0);
   \draw[thin, dashed, color=red] ({1/sqrt(3)},{-1/sqrt(3)},0) --++
   ({-1/sqrt(3)},0,0);
   \draw[thin, dashed, color=red] (0, 0, {-1/sqrt(3)}) --++
   ({1/sqrt(3)}, {-1/sqrt(3)}, 0);
   \draw[thin, dashed, color=red] ({1/sqrt(3)},{-1/sqrt(3)},0) --++
   ({-1/sqrt(3)},{1/sqrt(3)},0);
   \draw[thin, dashed, color=blue] (P_2) --++ (0,0,{1/sqrt(3)});
   \draw[thin, dashed, color=blue] ({-1/sqrt(3)},{1/sqrt(3)},0) --++
   (0,{-1/sqrt(3)},0);
   \draw[thin, dashed, color=blue] ({-1/sqrt(3)},{1/sqrt(3)},0) --++
   ({1/sqrt(3)},0,0);
   \draw[thin, dashed, color=blue] (0, 0, {-1/sqrt(3)}) --++
   ({-1/sqrt(3)}, {1/sqrt(3)}, 0);
   \draw[thin, dashed, color=blue] ({-1/sqrt(3)},{1/sqrt(3)},0) --++
   ({1/sqrt(3)},{-1/sqrt(3)},0);
   \draw[thin, dashed, color=purple] (P_3) --++ (0,0,{-1/sqrt(3)});
   \draw[thin, dashed, color=purple] ({-1/sqrt(3)},{-1/sqrt(3)},0) --++
   (0,{1/sqrt(3)},0);
   \draw[thin, dashed, color=purple] ({-1/sqrt(3)},{-1/sqrt(3)},0) --++
   ({1/sqrt(3)},0,0);
   \draw[thin, dashed, color=purple] (0, 0, {1/sqrt(3)}) --++
   ({-1/sqrt(3)}, {-1/sqrt(3)}, 0);
   \draw[thin, dashed, color=purple] ({-1/sqrt(3)},{-1/sqrt(3)},0) --++
   ({1/sqrt(3)},{1/sqrt(3)},0);
   % Axes in 3 d coordinate system
   \draw[-stealth] (0,0,0) -- (1.80,0,0)
       node[below left] {$D_1$};
   \draw[-stealth] (0,0,0) -- (0,1.30,0)
       node[below right] {$D_2$};
   \draw[-stealth] (0,0,0) -- (0,0,1.30)
       node[above] {$D_3$};
   %
   \draw[dashed, gray] (0,0,0) -- (-1,0,0);
   \draw[dashed, gray] (0,0,0) -- (0,-1,0);
   \draw[dashed, gray] (0,0,0) -- (0,0,-1);
   %
   \draw[dashed] ({1/sqrt(3)},{1/sqrt(3)},0) -- (1,1,0);
   \draw[dashed, color=purple] ({-1/sqrt(3)},{-1/sqrt(3)},0) -- (-1,-1,0);
   % Line from the origin to (P)
   \draw[thick, -stealth] (0,0,0) -- (P) node[right] {$x$};
   \draw[thick, -stealth, color=red] (0,0,0) -- (P_1) node[left] {$\rho _1(x)$};
   \draw[thick, -stealth, color=blue] (0,0,0) -- (P_2) node[right] {$\rho _2(x)$};
   \draw[thick, -stealth, color=purple] (0,0,0) -- (P_3) node[left] {$\rho_2(\rho_1(x))=\rho_1(\rho_2(x))$};
   % Add small circle at (P)
   \draw[fill = lightgray!50] (P) circle (0.5pt);
   \draw[fill = lightgray!50] (P_2) circle (0.5pt);
   \draw[fill = lightgray!50] (P_3) circle (0.5pt);
   \draw[fill = lightgray!50] (P_1) circle (0.5pt);
   \pgfmathsetmacro{\phivec}{225}
   \tdplotdrawarc[-stealth, color=purple]{(O)}{1}{45}{\phivec}{anchor=west}{$\pi$}
   \end{tikzpicture}
 \end{center}
 \item Pour l'implication directe on va montrer dans un premier temps que pour une rotation  vectorielle $\rho$ distincte de $\mathrm{Id_E}$, d'axe $D=\mathbb{R}\omega$ où $\omega$ est un vecteur unitaire, si $\Delta$ une droite vectorielle distincte de $D$ et telle que $\rho(\Delta)=\Delta$, alors $D$ et $\Delta$ sont orthogonales et $\rho$ est une demi-tour d'axe $D$. Et on conclut en utilisant le fait que les droites de rotation de $\rho_1$ et $\rho_2$ sont soit egales soit orthogonales et invariantes par ces rotations.\\
 \enskip

 Soient donc $\rho$, $D$ et $\Delta$ comme si dessus, et soient $e_D$ et $e_\Delta$ des vecteurs unitaires directeurs de $D$ et $\Delta$.
 On sait que $\rho(e_D)=e_D$ car $e_D$ est un vecteur directeur de $D$ l'axe de rotation de $\rho$, de plus $\rho(\Delta)=\Delta$ implique que :
 $$\exists \alpha \in \mathbb{R}, \rho(e_\Delta)=\alpha e_\Delta$$
 Et par orthogonalité de $\rho$, $\alpha=\pm 1$ car $e_\Delta$ est unitaire de plus, si $\alpha=1$ alors $D=\Delta$ ou $\rho=\mathrm{Id_E}$,
 donc $\alpha = -1$, alors par orthogonalité de $\rho$ : $$\langle e_D,e_\Delta  \rangle=\langle \rho(e_D),\rho(e_\Delta) \rangle=-\langle e_D,e_\Delta  \rangle = 0$$
 Et en complétant en une base orthonormée $B=(e_D,e_\Delta,e_3)$, la matrice de $\rho$ dans cette base est :
 \begin{center}
 \[
 \mathrm{P}=
 \begin{pmatrix}
 1 & 0 & *_1\\
 0 & -1 & *_2 \\
 0 & 0 & -1
 \end{pmatrix}
 \text{ (car $\mathrm{Det}(\rho)=1)$ }
 \]
 \end{center}
 Et puisque $~^tPP=P~^tP=I$ alors $*_1=*_2=0$ et donc :
 \begin{center}
 \[
 \mathrm{P}=
 \begin{pmatrix}
 1 & 0 & 0\\
 0 & -1 & 0 \\
 0 & 0 & -1
 \end{pmatrix}
 \]
 \end{center}
 On conclut donc que $D$ et $\Delta$ sont orthogonales et que $\rho$ est une demi-tour d'axe $D$.\par
 Si on prend maintenant $D = D_1$ et $\Delta = \rho_2(D_1)$, alors :
 \begin{align*}
   \Delta \neq D &\Rightarrow \text{$D$ et $\Delta$ sont orthogonales et $\forall x \in \Delta, \rho_1(x) = -x$}\\
   &\Rightarrow \rho_1(\rho_2(x))=-\rho_2(x), \forall x \in D\\
   &\Rightarrow \rho_1(\rho_2(x))=-\rho_2(\rho_1(x)) \neq \rho_2(\rho_1(x))
 \end{align*}
 On conclut que $\rho_2(D_1)=D_1$, de même $\rho_1(D_2)=D_2$, et donc on en déduit que $\rho_1$ et $\rho_2$ sont deux rotations d'axes orthogonaux.\par
\end{itemize}
\end{proof}
\tdplotsetmaincoords{60}{110}
\begin{Cor}
  Une condition necessaire que doivent satisfaire $\rho_1$ et $\rho_2$ pour que $\mathrm{G} = \langle\rho_1, \rho_2 \rangle$ soit libre, c'est qu'ils doivent ne pas avoir le même axe de rotation et ne pas être des demi-tour d'axes orthogonaux.
\end{Cor}
\noindent
Étudiant maintenant les groupes de la forme $\mathrm{G} = \langle\rho_1, \rho_2 \rangle$ où $\rho_1, \rho_2 \in \mathcal{SO}(3)$ qui ne commutent pas.\par
\begin{prop}\label{prop2}
  Soit $g \in \mathrm{G}-\{\mathrm{Id_E}\}$ alors :
  $$\exists n \in \mathbb{N}^*,  (s_1, ..., s_n) \in \{\rho_1, \rho_2\}^n,s_i \neq s_{i+1}, (a_1, ..., a_n) \in \mathbb{Z}^{*n}, g = s_1^{a_1}s_2^{a_2}...s_n^{a_n}$$
\end{prop}
\begin{proof}
  \hfill
\par
  On a $\mathrm{G}=\{s_1^{a_1}s_2^{a_2}...s_n^{a_n}, n\in \mathbb{N}^*, (s_1, ..., s_n) \in \{\rho_1, \rho_2\}^n, (a_1, ..., a_n) \in \mathbb{Z}^n\}$, alors si $g \in \mathrm{G}-\{\mathrm{Id_E}\}$ : $$\exists n \in \mathbb{N}^*,  (s_1, ..., s_n) \in \{\rho_1, \rho_2\}^n, (a_1, ..., a_n) \in \mathbb{Z}^{*n}, g = s_1^{a_1}s_2^{a_2}...s_n^{a_n}$$\par Il suffit donc de regrouper les éléments successifs lorsque $s_i=s_{i+1}$ et supprimer les termes pour lesquelle \par $a_i=0$ car dans ce cas $s_i^{a_i}=\mathrm{Id_E}$.

\end{proof}
% %%%%%%%%%%%%%%%%%%%%%%%%%%%%%%%%%%%%%%%%%%%%%%%%%%%%%%%%%%%%%%%%%%%%%%%%%%%%%%%%%%%%%%%%%%%%%%%%%%%%%%%%%%%%%%%%%%%%%%%%%
% 3. On suppose que $\rho_1$ et $\rho_2$ sont deux rotations d'angles respectifs $\alpha_1$ et $\alpha_2$ autour de la droite $D$ dérigée et orientée par le vecteur unitaire $\omega$.\par
% %%%%%%%%%%%%%%%%%%%%%%%%%%%%%%%%%%%%%%%%%%%%%%%%%%%%%%%%%%%%%%%%%%%%%%%%%%%%%%%%%%%%%%%%%%%%%%%%%%%%%%%%%%%%%%%%%%%%%%%%%
%
% a) Par définition on a $\mathrm{G}=\{s_1^{a_1}s_2^{a_2}...s_n^{a_n}, n\in \mathbb{N}^*, (s_1, ..., s_n) \in \{\rho_1, \rho_2\}^n, (a_1, ..., a_n) \in \mathbb{Z}^n\}$, et donc: $$\mathrm{G}=\{\rho_1^{n_1}\rho_2^{n_2} | (n_1, n_2)\in \mathbb{Z}^2\}=\mathrm{H}, \text{ car  $\rho_1$ et $\rho_2$ commutent}.$$
% %%%%%%%%%%%%%%%%%%%%%%%%%%%%%%%%%%%%%%%%%%%%%%%%%%%%%%%%%%%%%%%%%%%%%%%%%%%%%%%%%%%%%%%%%%%%%%%%%%%%%%%%%%%%%%%%%%%%%%%%%
% b) Si on suppose de plus que $x\alpha_1+y\alpha_2+z\pi=0\Rightarrow x=y=z=0$ alors :\\
% Soit $r\in\mathrm{G}$ alors d'après a) $\exists (n_1, n_2) \in \mathbb{Z}^2, r=\rho_1^{n_1}\rho_2^{n_2}$, de plus si $\exists (n_1', n_2') \in \mathbb{Z}^2, r=\rho_1^{n_1'}\rho_2^{n_2'}$ alors :
% \begin{align*}
%   \rho_1^{n_1}\rho_2^{n_2}=\rho_1^{n_1'}\rho_2^{n_2'} &\Rightarrow \rho_1^{n_1-n_1'}\rho_2^{n_2-n_2'} = \mathrm{Id_E} \text{ ($\rho_1$ et $\rho_2$ commutent)}\\
%    &\Rightarrow (n_1-n_1')\alpha_1 + (n_2-n_2')\alpha_2 = 2\pi z \text{ pour un certain $z$ dans $\mathbb{Z}$}\\
%    &\Rightarrow n_1=n_1' \text{ et } n_2 = n_2'
% \end{align*}
% Donc :$$\forall r \in \mathrm{G}, \exists! (n_1, n_2) \in \mathbb{Z}^2, r=\rho_1^{n_1}\rho_2^{n_2}$$.\par
% %%%%%%%%%%%%%%%%%%%%%%%%%%%%%%%%%%%%%%%%%%%%%%%%%%%%%%%%%%%%%%%%%%%%%%%%%%%%%%%%%%%%%%%%%%%%%%%%%%%%%%%%%%%%%%%%%%%%%%%%%
% 4. On suppose maintenat que $\rho_1$ et $\rho_2$ sont deux demi-tour d'axes orthogonaux alors ils commutent, de plus $\rho_1^2=\rho_2^2=\mathrm{Id_E}$ alors :
% $$
% r=\rho_1^{n_1}\rho_2^{n_2} = \left\{
% \begin{array}{ll}
%   \mathrm{Id_E} \text{ si $n_1$, $n_2$ sont pairs}\\
%   \rho_2 \text{ si $n_1$ pair, $n_2$ impair}\\
%   \rho_1 \text{ si $n_1$ impair, $n_2$ pair}\\
%   \rho_1\rho_2=\rho_2\rho_1 \text{ si $n_1$ impair, $n_2$ impair}
% \end{array}
% \right.
% $$
% Et la table du groupe $\mathrm{G}$ est :
% \begin{center}
% \begin{tabular}{|c|c|c|c|c|}
% \hline
% o & $\mathrm{Id_E}$ & $\rho_1$ & $\rho_2$ & $\rho_1\rho_2$\\
% \hline
% $\mathrm{Id_E}$ & $\mathrm{Id_E}$ & $\rho_1$ & $\rho_2$ & $\rho_1\rho_2$\\
% \hline
% $\rho_1$ & $\rho_1$ & $\mathrm{Id_E}$ & $\rho_1\rho_2$ & $\rho_2$ \\
% \hline
% $\rho_2$ & $\rho_2$ & $\rho_1\rho_2$ & $\mathrm{Id_E}$ & $\rho_1$  \\
% \hline
% $\rho_1\rho_2$ & $\rho_1\rho_2$ & $\rho_2$ & $\rho_1$ & $\mathrm{Id_E}$\\
% \hline
% \end{tabular}
% \end{center}
% \par
%%%%%%%%%%%%%%%%%%%%%%%%%%%%%%%%%%%%%%%%%%%%%%%%%%%%%%%%%%%%%%%%%%%%%%%%%%%%%%%%%%%%%%%%%%%%%%%%%%%%%%%%%%%%%%%%%%%%%%%%%
% 5. On suppose que $\rho_1$ et $\rho_2$ ne commutent pas.\par
% %%%%%%%%%%%%%%%%%%%%%%%%%%%%%%%%%%%%%%%%%%%%%%%%%%%%%%%%%%%%%%%%%%%%%%%%%%%%%%%%%%%%%%%%%%%%%%%%%%%%%%%%%%%%%%%%%%%%%%%%%
% a) Par définition on a $\mathrm{G}=\langle\rho_1, \rho_2 \rangle=\{s_1^{a_1}s_2^{a_2}...s_n^{a_n}, n\in \mathbb{N}^*, (s_1, ..., s_n) \in \{\rho_1, \rho_2\}^n, (a_1, ..., a_n) \in \mathbb{Z}^n\}$.\par
% %%%%%%%%%%%%%%%%%%%%%%%%%%%%%%%%%%%%%%%%%%%%%%%%%%%%%%%%%%%%%%%%%%%%%%%%%%%%%%%%%%%%%%%%%%%%%%%%%%%%%%%%%%%%%%%%%%%%%%%%%
% b) Soit $g \in \mathrm{G}-\{\mathrm{Id_E}\}$ alors : $$\exists n \in \mathbb{N}^*,  (s_1, ..., s_n) \in \{\rho_1, \rho_2\}^n, (a_1, ..., a_n) \in \mathbb{Z}^{*n}, g = s_1^{a_1}s_2^{a_2}...s_n^{a_n}$$
% Si on regroupe les éléments successifs lorsque $s_i=s_{i+1}$ et on supprime les termes pour lesquelle $a_i=0$ car dans ce cas $s_i^{a_i}=\mathrm{Id_E}$ alors :
% $$\boxed{\forall g \in \mathrm{G}-\{\mathrm{Id_E}\}, \exists n \in \mathbb{N}^*,  (s_1, ..., s_n) \in \{\rho_1, \rho_2\}^n,s_i \neq s_{i+1}, (a_1, ..., a_n) \in \mathbb{Z}^{*n}, g = s_1^{a_1}s_2^{a_2}...s_n^{a_n}}$$
\begin{remarkk}
  Cette écriture n'est en générale pas unique (si $\rho_1$ est une demi-tour alors $\rho_1=\rho_1^3$). Dans la partie suivante on va construire un exemple où cette décomposition sera unique, en d'autres mots on va construire un groupe libre de rang $2$.
\end{remarkk}
