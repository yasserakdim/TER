\chapter{Décomposition d'ensembles de points en parties respectivement congruentes.}
Dans ce chapitre on va étudier le paradoxe dans le cas général en s'appuyant sur les résultats du deuxième chapitre, et on
va montrer en fin du chapitre que le paradoxe n'est pas vrai en dimension inférieur à $2$ .
\section{Propriétés générales de l'équivalence par décomposition finie.}\label{3.}
% \setcounter{definition}{0}
\setcounter{lemma}{0}
\newtheorem{deff}{Définition}
\newtheorem{T}{Théorème}
\newtheorem{Co}{Corollaire}
\newtheorem{PR}{Proposition}
% \setcounter{theorem}{0}
% \setcounter{prop}{0}
Les raisonnements de ce chapitre sont valables pour les ensembles de points situés dans un espace quelconque, sur lequel est faite l'hypothèse unique qu'à tout couple de points $(a,b)$ correspond un nombre réel $\mathcal{\rho}(a,b)$ appelé distance des points $a$ et $b$. On va se baser dans ce chapitre sur les articles originaux de Stefan Banach et Alfred Tarski \cite{cite0}.
\begin{deff}\label{def3.1}
  On dit que les ensembles de points $\mathrm{A}$ et $\mathrm{B}$ sont congruents et on écrit $$\mathrm{A} \cong \mathrm{B}$$
  S'il existe une fonction $\varphi$ qui transforme d'une façon biunivoque $\mathrm{A}$ en $\mathrm{B}$, et qui vérifie  $$\forall a,a' \in \mathrm{A}, \mathcal{\rho}(a, a') =  \mathcal{\rho}\left(\varphi\left(a\right), \varphi\left(a'\right)\right)$$
\end{deff}

\begin{deff}\label{def3.2}
  On dit que les ensembles de points $\mathrm{A}$ et $\mathrm{B}$ sont équivalents par décomposition finie et on écrit :$$\mathrm{A} \sim \mathrm{B}$$ s'il existe un $n \in \mathbb{N}^*$ et des familles d'ensembles $\left(\mathrm{A}_i\right)_{1\le i \le n}$ et $\left(\mathrm{B}_i\right)_{1\le i \le n}$ tels que :
  \begin{center}
  \begin{enumerate}
    \item $\mathrm{A} = \underset{1\le i \le n}{\bigcup}\mathrm{A}_i \text{ et } \mathrm{B} = \underset{1\le i \le n}{\bigcup}\mathrm{B}_i$
    \item $\mathrm{A}_i \cap \mathrm{A}_j = \mathrm{B}_i \cap \mathrm{B}_j = \emptyset, \forall i\neq j$
    \item $\mathrm{A}_i \cong \mathrm{B}_i, \forall i=1, ...,n$
  \end{enumerate}
\end{center}
\end{deff}

\begin{PR}
  Si $\mathrm{A} \cong \mathrm{B}$ alors $\mathrm{A} \sim \mathrm{B}$
\end{PR}
\begin{proof}
  Ceci découle de la \hyperref[def3.2]{Définition 2}, en prenant $n = 1$.
\end{proof}
\begin{PR}
  La relation $\sim$ est une relation d'équivalence.
\end{PR}
\begin{proof}
  \hfill
  \begin{itemize}
    \item Réflexivité : $A$ est une partition de $A$ et $A=\mathrm{Id_E}(A)$, avec $\mathrm{Id_E}$ bijective de $\mathrm{A}$ dans $\mathrm{A}$ et vérifie :$$\forall a,a' \in \mathrm{A}, \mathcal{\rho}(a, a') =  \mathcal{\rho}\left(\mathrm{Id_E}\left(a\right), \mathrm{Id_E}\left(a'\right)\right)$$
    Donc :$$A\sim A$$.
    \item Symétrie : Si $A \sim B$ alors $\exists (A_i)_{1 \le i\le n}, (B_i)_{1 \le i\le n}, (\varphi_i)_{1 \le i\le n} \text{ tels que } B_i=\varphi_i(A_i)$ et donc :$$A_i=\varphi_i^{-1}(B_i), \forall i=1,..,n$$
    Avec $\forall i=1,..,n$, $\varphi_i^{-1}$ est bijective de $\mathrm{B}_i$ dans $\mathrm{A}_i$, et :
    $$\forall b=\varphi(a),b'=\varphi(a') \in \mathrm{B}_i, \mathcal{\rho}(b, b') = \mathcal{\rho}(\varphi(a), \varphi(a'))=\mathcal{\rho}(a, a') =\mathcal{\rho}\left(\varphi_i^{-1}\left(b\right), \varphi_i^{-1}\left(b'\right)\right)$$
     D'où  $B \sim A$.
    \item Transitivité : Si $A \sim B$ et $B \sim C$ alors, $\exists I, J$ finis, $\exists (A_i)_{i \in I}, (B_i)_{i \in I}, (\varphi_i)_{i \in I}$, $(B_j)_{j \in J}, (C_j)_{j \in J}$ et $(\varphi_j)_{j \in J}$ tels que $(A_i)_{i \in I}$ est une partition de $\mathrm{A}$, $(B_i)_{i \in I}$ et $(B_j)_{j \in J}$ sont des partitions de $B$ et $(C_j)_{j \in J}$ est une partition de $C$, et $\forall (i,j) \in I \times J$ :
    % $\exists I$ fini, $\exists (A_i)_{i \in I}, (B_i)_{i \in I}, (\varphi_i)_{i \in I}$  tels que $\forall i \in I$:\\
     $$\varphi_i \text{ est bijective de $\mathrm{A}_i$ dans $\mathrm{B}_i$, et } \forall a,a' \in \mathrm{A}, \mathcal{\rho}(a, a') =  \mathcal{\rho}\left(\varphi_i\left(a\right), \varphi_i\left(a'\right)\right)$$
     et :
     % et $\exists J$ fini $\exists (B_j)_{j \in J}, (C_j)_{j \in J}, (\varphi_j)_{j \in J}$  tels que  $\forall j \in J$:\\
          $$\varphi_j \text{ est bijective de $\mathrm{B}_j$ dans $\mathrm{C}_j$, et } \forall b,b' \in \mathrm{B}, \mathcal{\rho}(b, b') =  \mathcal{\rho}\left(\varphi_j\left(b\right), \varphi_j\left(b'\right)\right)$$

    On considère l'ensemble $K=\{ (i,j) \in I \times J \mid B_i\cap B_j\neq \emptyset \}$, et on considère :\\
    $$(B_i\cap B_j)_{(i,j)\in K} \text{, } (\varphi_i^{-1}(B_i\cap B_j))_{(i,j)\in K}\text{ et } (\varphi_j^{-1}(B_i\cap B_j))_{(i,j)\in K}$$ partitions de $B$, $A$ et $C$ respectivement.\\
     On a donc $\forall (i,j)\in K$,
     $\varphi_i^{-1} \circ \varphi_j$ est une bijection de $(\varphi_j^{-1}(B_i\cap B_j))_{(i,j)}$ dans $(\varphi_i^{-1}(B_i\cap B_j))_{(i,j)}$.\\ De plus $$\forall c=\varphi_j^{-1} \circ \varphi_i(a),c'=\varphi_j^{-1} \circ \varphi_i(a') \in (\varphi_j^{-1}(B_i\cap B_j))_{(i,j)}$$
     on a que
    $$\mathcal{\rho}(c, c') = \mathcal{\rho}(\varphi_j^{-1} \circ \varphi_i(a), \varphi_j^{-1} \circ \varphi_i(a'))=\mathcal{\rho}(\varphi_i(a), \varphi_i(a')) =\mathcal{\rho}\left(a,a'\right) = \mathcal{\rho}(\varphi_i^{-1} \circ \varphi_j(c), \varphi_i^{-1} \circ \varphi_j(c'))$$
    Il s'en  suit que $A \sim C$.
  \end{itemize}
\end{proof}
\begin{PR}\label{p3}
   Soient $\left(\mathrm{A}_i\right)_{1\le i \le n}$ et $\left(\mathrm{B}_i\right)_{1\le i \le n}$ deux familles d'ensembles disjoints, tels que :
   $$\mathrm{A}_i \sim \mathrm{B}_i, \forall i=1,..,n$$
   Alors : $$\underset{1\le i \le n}{\cup} \mathrm{A}_i \sim \underset{1\le i \le n}{\cup} \mathrm{B}_i$$
\end{PR}
\begin{proof}
Il suffit de le prouver pour $n=2$ et on généralise le résultat par récurrence.\\ Soient donc $A_1$, $A_2$, $B_1$ et $B_2$ tels que $A_1 \sim B_1$ et $A_2 \sim B_2$.\\
Alors $\exists I, J$ et des partitions finies de $A_1$, $A_2$, $B_1$, $B_2$ :
$((A_{1_i})_{i \in I})$, $((B_{1_i})_{i \in I})$, $((A_{2_j})_{j \in J})$, $((B_{2_j})_{j \in J})$ tels que :
$$\forall (i,j) \in I\times J, A_{1_i} \cong B_{1_i} \text{ et } A_{2_j} \cong B_{2_j}$$
Et puisque $(A_1, A_2)$ est une partition de $A_1\cup A_2$, et $(B_1, B_2)$ est une partition de $B_1\cup B_2$, alors :\\
$\left(\underset{i}{\bigcup} A_{1_i} \right) \cup \left(\underset{j}{\cup} A_{2_j}\right)$ et $\left(\underset{i}{\bigcup} B_{1_i} \right) \cup \left(\underset{j}{\cup} B_{2_j}\right)$ sont des partitions de  $A_1\cup A_2$, et $B_1\cup B_2$ donc : $$A_1 \cup A_2 \sim B_1 \cup B_2$$
Le résultat se généralise par récurrence pour toute collection de famille finies d'ensembles de points.
\end{proof}
\begin{PR}\label{p4}
  Si $\mathrm{A} \sim \mathrm{B}$ alors il existe une bijection $\psi$ de $A$ dans $B$ tel que pour tout sous-ensemble $C$ de $A$ on a que $C \sim \psi(C)$.
\end{PR}

\begin{proof}
  \hfill

\noindent
  On suppose que $A \sim B$, et soient $\left(\mathrm{A}_i\right)_{1\le i \le n}$, $\left(\mathrm{B}_i\right)_{1\le i \le n}$ et $\left(\varphi_i\right)_{1\le i \le n}$ qui vérifient les hypothèses de la définition \par de l'équivalence par décomposition finie. On pose :
  \begin{align*}
  \psi \colon A &\to B\\
  x &\mapsto \varphi_i(x), \text{ avec $i$ tel que $x \in A_i$}.
  \end{align*}
  Puisque $\forall i \in I, B_i = \varphi_i(A_i)$ alors la fonction est clairement surjective, de plus si $\psi(x)= \psi (y)$, alors $\exists i, j$ tel que $\varphi_i(x)=\varphi_j(y)$, si $i\ne j$ alors $\varphi_i(x) \in B_i$ et $\varphi_j(y) \in B_j$ et $B_i \cap B_j = \emptyset$ donc $\varphi_i(x)\neq \varphi_j(y)$ et donc $i=j$, et par injectivité de $\varphi_i$ on conclut que $x=y$, et alors $\psi$ est injective, donc bijective.\\
  Pour tout sous-ensemble $C$ de $A$, on a $(C\cap A_i)_{i \in I}$ est une partition de $C$ et $(\psi(C) \cap B_i)_{i \in I}$ est une partition de $\psi(C)$, et $\varphi_i(C\cap A_i)= \varphi_i(C)\cap B_i= \psi(C\cap A_i)=\psi(C)\cap B_i$ et donc $\forall i$:
   $$C\cap A_i \cong  \psi(C)\cap B_i$$ et alors $C \sim \psi(C)$.
\end{proof}

\begin{Co}
  Soit $\left(\mathrm{A}_i\right)_{1\le i \le n}$ une partition de $\mathrm{A}$ et soit B tel que $\mathrm{A} \sim \mathrm{B}$, alors il existe une partition $\left(\mathrm{B}_i\right)_{1\le i \le n}$ de $\mathrm{B}$ telle que $\mathrm{A}_i \sim \mathrm{B}_i, \forall i=1,..,n$.
\end{Co}
\begin{proof}
  \hfill

Il suffit de prendre $\mathrm{B}_i = \psi(\mathrm{A}_i)$ où $\psi$ est la fonction vérifiant les hypothèses de la proposition\par précédente, alors :
\begin{itemize}
  \item $\mathrm{A}_i \sim \mathrm{B}_i, \forall i=1,..,n$
  \item $(\mathrm{B}_i)_{1\le i \le n} = (\psi(\mathrm{A}_i))_{1\le i \le n}$ est une partition de $\mathrm{B}$ (en utilise le même raisonnement que dans \hyperref[lemme 2]{corollaire 1 du préliminaires}).
\end{itemize}
\end{proof}
\begin{Co}
 Si $\mathrm{A} \sim \mathrm{B}$, alors pour tout ensemble $\mathrm{C}$ dans $\mathrm{A}$, il existe un ensemble $\mathrm{D}$ dans $\mathrm{B}$ tel que :
 $$\mathrm{C} \sim \mathrm{D} \text{ et si }  \mathrm{C} \neq \mathrm{A} \text{, alors } \mathrm{D} \neq \mathrm{B}$$
\end{Co}
\begin{proof}
Il suffit encore de prendre $D = \psi(C)$ alors :
\begin{itemize}
  \item $\mathrm{C} \sim \mathrm{D}$
  \item Si $\mathrm{C} \neq \mathrm{A}$ alors $\mathrm{D}=\psi(C) \neq \mathrm{B}$, puisque $\psi$ est bijective.
\end{itemize}
\end{proof}

\begin{T}(Théorème de Banach-Schröder-Bernstein)\label{T1}
  Soient $\mathrm{A}_1 \subset \mathrm{A}$ et $\mathrm{B}_1 \subset \mathrm{B}$ tel que $\mathrm{A} \sim \mathrm{B}_1 \text{ et } \mathrm{B} \sim \mathrm{A}_1$, alors on a que $\mathrm{A} \sim \mathrm{B}$.
\end{T}
\noindent
Avant de démontrer ce théorème, énonçons un autre théorème qui va nous servir dans la démonstration.
\begin{T}\label{T2}
  Soit $\varphi$ une fonction qui transforme de façon biunivoque l'ensemble $A$ en un sous ensemble de $B$, et $\psi$ une fonction qui transforme de façon biunivoque un sous ensemble de $A$ en l'ensemble $B$. Alors il existe une partition $(A_1, A_2)$ de $A$ et une partition $(B_1, B_2)$ de $B$ qui satisfait
  $$\varphi(A_1) = B_1 \text{ et } \psi(A_2) = B_2$$
\end{T}
\begin{proof}
  Soit $a$ un élément de $A$, et soit $C(a)$ l'intersection de tous les sous-ensembles $X$ de $A$ qui vérifient
  \begin{itemize}
    \item $a \in X$.
    \item Si $x\in X$ on a $\psi^{-1} \circ \varphi(x) \in X$.
    \item Et si $x\in X$ et $\varphi^{-1} \circ \psi(x)$ existe, on a $\varphi^{-1} \circ \psi(x) \in X$.
  \end{itemize}
  Les ensemble $C(a)$ où $a$ dans $A$ sont soit disjoints soit égaux. Soit ensuite $$A_2 =  \{a \in A \mid C(a) \in \psi^{-1}(B)\}$$
  et soit $A_1$ le complémentaire de $A_2$ c'est à dire $$A_1 = A-A_2$$ Soient ensuite $$B_1 = \varphi(A_1) \text{ et } B_2 = \psi(A_2)$$
  on peut faire ça car $A_2 \subset \psi^{-1}(B)$.\\
  On a alors $B_1 \cup B_2 \subset B$, montrons l'inclusion réciproque.
  Soit $b \in B$, et $a = \psi^{-1}(b)$ alors on a que
  \begin{itemize}
    \item Si $a \in A_2$ alors $b \in \psi(A_2) = B_2$.
    \item Si $a \in A_1$ alors $\varphi^{-1}(b)$ existe, car sinon $a \in \psi^{-1}(B)$ et $C(a) \subset \psi^{-1}(B)$ ce qui signifie que $a\in A_2$ ce qui est absurde car $A_1 \cap A_2 = \emptyset$. De plus $\varphi^{-1}(b)$ existe et appartient à $A_1$, car $C(\psi^{-1}(b)) = C(\varphi^{-1}(b))$ car $\psi^{-1}(b) = \psi^{-1}\circ \varphi \circ \varphi^{-1}(b)$, et $\psi^{-1}(b) \in A_1$.
  \end{itemize}
  On conclut donc que $B \subset B_1 \cup B_2$ et alors $B = B_1 \cup B_2$
\end{proof}
\begin{proof}[Démonstration du Théorème 1]
  \hyperref[T2]{Théorème 2} et \hyperref[p4]{Proposition 4} affirments qu'on peut trouver une partition $(A_1, A_2)$ de $A$ et une partition $(B_1, B_2)$ de $B$ tels que $A_1 \sim B_1$ et $A_2 \sim B_2$, alors $A_1 \cup A_2 \sim B_1 \cup B_2$ et alors $A \sim B$.
\end{proof}
\begin{Co}\label{c2}
  Si $\mathrm{C} \subset \mathrm{B} \subset \mathrm{A}$ et $\mathrm{C} \sim \mathrm{A}$, alors :
  $$\mathrm{A} \sim \mathrm{B} \text{ et } \mathrm{B} \sim \mathrm{C}$$
\end{Co}
\begin{proof}
On a $\mathrm{B} \sim \mathrm{B} \subset \mathrm{A}$ et $\mathrm{A} \sim \mathrm{C} \subset \mathrm{B}$ donc d'après le \hyperref[T1]{Théorème 1}, en prenant $\mathrm{A}_1 = \mathrm{C}$ et $\mathrm{B}_1=\mathrm{B}$, on trouve que $\mathrm{A} \sim \mathrm{B}$, et puisque $\mathrm{A} \sim \mathrm{C}$ alors on a aussi $\mathrm{B} \sim \mathrm{C}$ (car $\sim$ est une relation d'équivalence).
\end{proof}
\begin{PR}\label{pr5}
  Si $\mathrm{A} \sim \mathrm{A} \cup \mathrm{B}_i \text{ pour }i=1,..,n $ alors :
  $$\mathrm{A} \sim \mathrm{A} \cup \left(\underset{1\le i \le n}{\cup}\mathrm{B}_i\right)$$
\end{PR}
\begin{proof}
  \hfill
  \begin{itemize}
    \item  On va commencer par le cas où $\mathrm{A},\mathrm{B}_1,...,\mathrm{B}_n$ sont disjoints deux à deux.\\
     On va procéder par récurrence. La proposition est triviale pour $n=1$, supposons qu'elle est vraie pour un certain $n \in \mathbb{N}$, et montrons le pour $n+1$. On a alors :
     \begin{enumerate}
       \item $\mathrm{A} \sim \mathrm{A} \cup \left(\underset{1\le i \le n}{\cup}\mathrm{B}_i\right)$.
       \item $\mathrm{A} \sim \mathrm{A} \cup \mathrm{B}_{n+1}$.
       \item Les ensembles $\mathrm{A},\mathrm{B}_1,...,\mathrm{B}_n, \mathrm{B}_{n+1}$ sont disjoints.
     \end{enumerate}
     On a alors d'après la \hyperref[p3]{proposition 3}, $1.$ et $3.$ impliquent que :
    \begin{enumerate}
      \setcounter{enumi}{3}
      \item $\mathrm{A} \cup \mathrm{B}_{n+1} \sim \mathrm{A} \cup \left(\underset{1\le i \le n+1}{\cup}\mathrm{B}_i\right)$
    \end{enumerate}
    Et puisque $\sim$ est une relation d'équivalence alors d'après $2.$ et $4.$ on a : $$\mathrm{A} \sim \mathrm{A} \cup \left(\underset{1\le i \le n+1}{\cup}\mathrm{B}_i\right)$$
    Ce qui termine la démonstration dans le cas où $\mathrm{A},\mathrm{B}_1,...,\mathrm{B}_n$ sont disjoints deux à deux.
    \item On va passer maintenant au cas général, et on pose :
    $$\mathrm{C}_1 = \mathrm{B}_1-\mathrm{A}, \text{ et } \mathrm{C}_k = \mathrm{B}_k - \left(\mathrm{A} \cup \left(\underset{1\le i \le k-1}{\cup}\mathrm{B}_i\right)\right), \text{ pour } k=2,..,n$$
    Il est clair que $\mathrm{A} \cup \left(\underset{1\le i \le k-1}{\cup}\mathrm{C}_i\right) = \mathrm{A} \cup \left(\underset{1\le i \le k-1}{\cup}\mathrm{B}_i\right)$, et $\mathrm{A} \subset \mathrm{A} \cup \mathrm{C}_k \subset \mathrm{A} \cup \mathrm{B}_k$ pour tout $k$.\\
    Le \hyperref[c2]{corollaire 3}, et le fait que $\mathrm{A} \sim \mathrm{A} \cup \mathrm{B}_k$, pour $k=1,..,n$, impliquent que $\mathrm{A} \sim \mathrm{A} \cup \mathrm{C}_k$, pour $k=1,..,n$.\\
    Et puisque les ensembles $\mathrm{A},\mathrm{C}_1,...,\mathrm{C}_n$ sont disjoints deux à deux par définition, alors on se ramène au premier cas, et on conclut donc la proposition est vraie.
  \end{itemize}
\end{proof}

\newtheorem{TT}{Théorème}
\newtheorem{PP}{Proposition}

\section{Retour à l'espace euclidien à 3 dimensions.}
\noindent
Les raisonnements de cette partie concernent l'espace euclidien à $3$ dimensions.
% \subsection{Rappels :}
On rappele le théorème de Hausdorff :
\begin{TT}
  \hfill

  Toute sphère $\mathrm{S}$ est équidécomposable à deux copies d'elle même. En d'autres termes, si $\Sigma_1$ et $\mathrm{S}$ sont deux sphères congruentes, alors : $\mathrm{S} \sim \mathrm{S} \cup \Sigma_1$.
\end{TT}

\begin{PP}\label{pp}
  Soit $A$ un ensemble borné d'un espace euclidien.
  Si $A$ contient une sphère $S$ alors $A \sim S$.
\end{PP}
\begin{proof}
  Puisque $A$ est borné alors on peut le décomposer : $A = \underset{1\le i \le n}{\bigcup}\mathrm{A}_i$ tel que chaque $A_i$ est inclut dans une sphère $S_i$ congruente à $S$.\\
  Alors on a $\mathrm{S} \sim \mathrm{S} \cup \mathrm{S}_i$ pour tout $i=1,...,n$, et donc d'après la \hyperref[pr5]{proposition 5}, $\mathrm{S} \sim \mathrm{S} \cup \left(\underset{1\le i \le n}{\cup}\mathrm{S}_i\right)$.\\
  Et puisque $S \subset A \subset S \cup \left(\underset{1\le i \le n}{\cup}\mathrm{S}_i\right)$, alors d'après le \hyperref[c2]{corollaire 3}, on conclut que $A \sim S$.
\end{proof}
\begin{TT}
  Soient $A$ et $B$ deux ensembles bornés quelconques d'intérieurs non vides, situés dans un espace euclidien de dimension 3, alors $A \sim B$.
\end{TT}
\begin{proof}
  Soient $S_1 \subset A$ et $S_2 \subset B$, tels que $S_1 \sim S_2$, (on peut trouver de telles sphères puisque $A$ et $B$ sont d'intérieurs non vides). Alors d'après la \hyperref[pp]{proposition 1}, $S_1 \sim A$ et $B \sim S_2$, et donc puisque $\sim$ est une relation d'équivalence on conclut directement que $\mathrm{A} \sim \mathrm{B}$.
\end{proof}

\section{L'espace euclidien à 1 ou 2 dimensions .}
\newtheorem{defff}{Définition}
\newtheorem{TTT}{Théorème}
\newtheorem{Coo}{Corollaire}
\newtheorem{PRR}{Proposition}
\newtheorem{rr}{Remarque}
\newtheorem{lm}{Lemme}
\noindent
Les raisonnements de cette partie concernent l'espace euclidien à $1$ ou $2$ dimensions.

\begin{TTT}
  Soient $A$ et $B$ deux ensembles linéaires ou plans bornés et mésurables au sens de Lebesgue, et soit $\mu$ la mesure de Lebesgue. On a alors que
  $$A \sim B \Rightarrow \mu(A) = \mu(B)$$
\end{TTT}
\begin{proof}
  On admet pour l'instant (à faire), que pour tout ensemble borné $A$ d'un espace euclidien à $1$ ou $2$ dimensions, il existe un nombre réel positif $f(A)$ qui vérifie
  \begin{enumerate}[label=(\roman*)]
    \item Si $A \cong B$ alors $f(A) = f(B)$.
    \item Si $A \cap B = \emptyset$ alors on a $f(A\cup B) = f(A)+f(B)$.
    \item Si $A$ est mésurable au sens de Lebesgue alors on a que $\mu(A) = f(A)$.
  \end{enumerate}
  Par définition de l'équivalence par décomposition finie, il existe un $n \in \mathbb{N}^*$ et des familles d'ensembles $\left(\mathrm{A}_i\right)_{1\le i \le n}$ et $\left(\mathrm{B}_i\right)_{1\le i \le n}$ tels que :
  \begin{center}
  \begin{enumerate}
    \item $\mathrm{A} = \underset{1\le i \le n}{\bigcup}\mathrm{A}_i \text{ et } \mathrm{B} = \underset{1\le i \le n}{\bigcup}\mathrm{B}_i$
    \item $\mathrm{A}_i \cap \mathrm{A}_j = \mathrm{B}_i \cap \mathrm{B}_j = \emptyset, \forall i\neq j$
    \item $\mathrm{A}_i \cong \mathrm{B}_i, \forall i=1, ...,n$
  \end{enumerate}
\end{center}
On trouve alors par $(i)$ et $3.$ que $\forall i=1,...,n$ on a que $f(A_i) = f(B_i)$, $\forall i=1,...,n$. De plus par $2.$ et par récurrence on conclut que $$f(A) = \sum_{i=1}^n f(A_i) = \sum_{i=1}^n f(B_i) = f(B)$$
Et puisque les ensembles $A$ et $B$ sont mésurables au sens de Lebesgue, alors d'après $(iii)$ on a que $\mu(A) = f(A)$ et $\mu(B) = f(B)$, et alors $$\mu(A) = \mu(B)$$
\end{proof}
\begin{rr}
  L'implication réciproque dans le théorème n'est pas vraie dans le cas général, mais elle l'est dans le cas des polygones.
\end{rr}
\noindent
Avant de démontrer l'implication réciproque pour le cas des polygones, on va énoncer un lemme qui va nous servir dans la démonstration.
\begin{lm}
Soit $A$ un ensemble plan qui n'est pas d'intérieur vide, et soit $B$ un ensemble somme d'un nombre fini de segments, on a alors que $$A \sim A \cup B$$
\end{lm}
\begin{proof}
  \hfill
  \begin{itemize}
    \item   Commençant par le cas où $A$ et $B$ sont disjoints. Puisque $A$ est d'intérieur non vide, on peut trouver un disque $C$ inclu dans $A$. De plus on peut décomposer $B$ en un nombre fini de segments $$B = \bigcup_{i=1}^n B_i$$
      tel que tous les segments sont de longueur inférieur au rayon de $C$.\\
      Soit $k \in [1,n]$ et soit $D_1$ un segment congruent à $B_k$ et situé sur un rayon de $C$ mais ne contenant pas le centre de $C$, et soit $\alpha$ un angle qui n'est pas de la forme $r\pi$ où $r\in \mathbb{Q}$.\\
      Par récurrence, soit $D_{n+1}$ le segment obtenu après rotation d'angle $n\alpha$ le segment $D_1$. On pose
      $$E = \bigcup_{i=1}^\infty D_n, F =  \bigcup_{i=2}^\infty D_n\text{ et } G = A-E$$
      On a alors que $$A = G \cup F \cup D_1 \text{ et } A \cup B_k = G \cup E \cup B_k$$
      alors puisque $F$ s'obtient en composant $E$ par la rotation d'angle $\alpha$ alors on a que $$G \cong G, F \cong E \text{ et } B_k \cong D_1$$
      De plus l'angle $\alpha$ n'est pas de la forme $r\pi$ où $r\in \mathbb{Q}$ donc la décomposition précédente de $A$ et $A \cup B_k$ est disjointe, on a alors que $$A \sim A \cup B_k$$
      et donc d'après le \hyperref[pr5]{Proposition 5}, on a que $$A \sim A \cup B$$
    \item Pour le cas général, on considère l'ensemble $A-B$ qui n'est pas d'intérieur vide, puisque $A-B$ et $B$ sont disjoints, alors par ce qui précède $$A-B \sim (A-B) \cup B = A \cup B$$
    de plus on a que $$A-B \subset A \subset A\cup B$$
    donc d'après le \hyperref[c2]{corollaire 3} on conclut que $$A \sim A \cup B$$
  \end{itemize}

\end{proof}
\begin{TTT}
  Si deux polygones $A$ et $B$ ont même aire alors on a que $A \sim B$.
\end{TTT}
\begin{proof}
  Puisque les polygones $A$ et $B$ ont le même aire, alors d'après \hyperref[wbg]{le Théorème de Wallace-Bolyai-Gerwien} on peut les décomposer en un nombre fini et égale de polygones respectivement congruents sans point intérieur communs. Soient donc $A_1, ..., A_n$ et $B_1,...,B_n$ les intérieurs de ses polygones, on a alors que $$\bigcup_{i=1}^n A_i \sim \bigcup_{i=1}^n B_i$$
  et puisque les ensemble $A - \bigcup_{i=1}^n A_i$ et $B -\bigcup_{i=1}^n B_i$ se composent d'un nombre fini de segments,  alors d'après le lemme précédent on a que $$\bigcup_{i=1}^n A_i \sim \bigcup_{i=1}^n A_i \bigcup \left(A - \bigcup_{i=1}^n A_i\right)=A \text{, et de même } \bigcup_{i=1}^n B_i \sim B$$
  Il s'en suit que $$A \sim B$$
\end{proof}
\begin{Coo}
  Deux polygones $A$ et $B$ sont équivalents par décomposition finie si et seulement si ils ont le même aire.
\end{Coo}


