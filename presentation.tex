\chapter{Présentation du projet}

En $1914$ \cite{cite3}, le mathématicien Hausdorff, un des fondateurs de la topologie générale, avait posé le problème suivant : est-il possible d'associer à tout ensemble borné $A \subset \mathbb{R}^n$  un nombre positif ou nul $\mu(A)$ tel que :
\begin{itemize}
  \item $\mu$ n'est pas pas identiquement nul.
  \item Si $A_1 \cap A_2 = \emptyset$, alors $\mu(A_1 \cup  A_2) = \mu(A_1) + \mu(A_2)$
  \item $\mu(A) = \mu(B)$ si $A$ et $B$ sont égaux à une rotation ou translation près.
\end{itemize}
En d'autre termes, il voudrait savoir si une mesure universelle invariante par rotation et translation peut exister.\par
Banach a construit une telle application $A \rightarrow \mu(A)$ pour $n = 1, 2$\cite{cite4}, et Hausdorff avait démontré par l'absurde l'impossibilité de ce problème pour $n \ge 3$ en établissant le paradoxe de la sphère qui dit que la sphère est \hyperref[ed]{équidécomposable} en deux copies d'elle même. Ainsi apparaît, à propos de ce problème, une différence fondamentale et intuitivement inexplicable entre le plan $(n=2)$ et l'espace usuel $(n=3)$.\par Analysant de plus près l'impossibilité du problème de la mesure universelle pour $n \ge 3$, Banach, dans un travail commun avec Tarski\cite{cite0}, démontre en $1923$ l'étrange résultat connu sous le nom de « paradoxe » de Banach-Tarski et qui est célèbre sous la forme donnée par Hausdorff : une boule dans $\mathbb{R}^3$
peut être divisée en un nombre fini de morceaux (non mesurables au sens de Lebesgue) avec lesquels on
peut reconstruire deux boules de la même taille. L’énoncé initial de Banach-Tarski est bien
plus fort : dans $\mathbb{R}^n$, $(n\ge 3)$ tous les ensembles $A$ et $B$ d’intérieurs bornés et non vides sont
équi-décomposables: en d’autres termes, on peut décomposer $A$ en un nombre fini de
morceaux à partir desquels on peut reconstruire $B$ par déplacements c'est à dire des rotations et des translations.\par
La différence fondamentale entre le plan et l'espace se traduit ainsi par des résultats du type suivant : deux polygones arbitraires dont l'un est contenu dans l'autre ne sont jamais équivalents par décomposition finie, alors que deux polyèdres arbitraires le sont toujours!\par
Le but principal du TER est de comprendre la démonstration du paradoxe de Banach-Tarski à partir des articles originaux de Stefan Banach et Alfred Tarski, on va donc démontrer dans le deuxième chapitre le paradoxe de la sphere de Hausdorff, puis dans le troisième chapitre on va étudier le paradoxe dans le cas général en s'appuyant sur les résultats du deuxième chapitre, et on va montrer que le paradoxe n'est pas vrai en dimension inférieur à $2$.



% \section{Sujet}
%
% Le paradoxe de Banach-Tarski est célèbre sous la forme suivante : une boule dans $\mathbb{R}^3$
% peut être divisée en un nombre fini de morceaux (non mesurables) avec lesquels on
% peut reconstruire deux boules de la même taille. L’énoncé initial du paradoxe est bien
% plus fort : dans $\mathbb{R}^n$, tous les ensembles $A$ et $B$ d’intérieurs bornés et non vides sont
% équi-décomposables, en d’autres termes, on peut décomposer $A$ en un nombre fini de
% morceaux à partir desquels on peut reconstruire $B$.
